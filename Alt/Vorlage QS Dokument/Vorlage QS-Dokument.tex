\documentclass[accentcolor=tud0b,12pt,paper=a4]{tudreport}

\usepackage[utf8]{inputenc}
\usepackage{ngerman}
\usepackage{parcolumns}
\usepackage{hyperref}

\newcommand{\titlerow}[2]{
	\begin{parcolumns}[colwidths={1=.15\linewidth}]{2}
		\colchunk[1]{#1:} 
		\colchunk[2]{#2}
	\end{parcolumns}
	\vspace{0.2cm}
}

\title{Entwicklung und Implementierung eines Plugins zur Ablaufsteuerung von Moodle-Tests}
\subtitle{Qualitätssicherungsdokument}
\subsubtitle{%
	\titlerow{Gruppe 22}{%
		Dominik Wittenberg <dominik.wittenberg@stud.tu-darmstadt.de>\\
		Paul Youssef <paul.youssef@stud.tu-darmstadt.de;>\\
		Pavel Azanov <pavel\_andreevic.azanov@stud.tu-darmstadt.de;>\\
		Robin Voigt <robin.voigt@stud.tu-darmstadt.de;>\\
		Alessandro Noli <alessandro\_fabio.noli@stud.tu-darmstadt.de>}
	\titlerow{Teamleiter}{Julia Hofmann <juliahofmann92@web.de>}
	\titlerow{Auftraggeber}{%
		Marcel Schaub <schaub@mathematik.tu-darmstadt.de>\\
		Arbeitsgruppe Didaktik\\
		Fachbereich Mathematik}
	\titlerow{Abgabedatum}{xx.xx.xxxx}}
\institution{Bachelor-Praktikum SoSe~2015\\Fachbereich Informatik}

\begin{document}

	\maketitle
	\tableofcontents 
	
	\chapter{Einleitung}
Am Fachbereich Mathematik der TU-Darmstadt wird mit der Plattform „Moodle“ gearbeitet, welches ein freies objektorientiertes Kursmanagementsystem und eine Lernplattform für Studierende von Universitäten und Hochschulen ist, welche im Jahre 2002 erschien und bis heute kontinuierlich weiterentwickelt worden ist. 
\\
Auf dieser Plattform können Studierende zu verschiedenen Veranstaltungen mathematische Aufgaben zu unterschiedlichen Themen lösen, um ihren Wissensstand auf einem Gebiet zu überprüfen. Bisher existieren Tests mit Fragen verschiedener Art, wie Matheaufgaben oder Multiple Choice. Diese Fragen müssen die Studierenden nacheinander in vorgegebener, statischer Reihenfolge beantworten. Am Ende des Tests wird ein Standard Feedback zur Bewertung der Aufgaben ausgegeben. 
Um den Studierenden zu helfen und sie besser durch solch einen Test bzw. die Aufgaben durchzuführen, möchte der Auftraggeber das adaptive Testen einführen. Dadurch beeinflusst die Antwort der vorherigen Frage die Frage, die im Anschluss gestellt wird. So kann z.B. nach einer falsch beantworteten Frage eine leichtere Frage zur selben Aufgabe oder nur nach den ersten Schritten der Aufgabe gefragt werden. So würde  der Studierende nach und nach zu der Lösung geführt und lernt dabei mit. Außerdem soll am Ende ein kumuliertes und adaptives Feedback das bisherige Standard Feedback ersetzen, welches dem Studierenden mehr Hilfe geben soll.
\\
\\
Ziel ist es nun, ein Plug-In zur Ablaufsteuerung von solchen adaptiven Moodle-Tests zu entwickeln und zu implementieren. Dadurch können zu einzelnen Aufgaben Unteraufgaben gestellt werden, die der Studierende beantworten muss, falls er eine falsche Antwort auf die eigentliche Frage gegeben hat. Erst nach Beantwortung aller Unteraufgaben gelangt dieser wieder zu eigentlichen nächsten Aufgabe. Diese Unteraufgaben können weitere Abzweigungen besitzen, die nach einer vom Ersteller festgelegten Reihenfolge abgefragt werden. Durch diesen \textbf{schleifen} \textbf{artigen} \textbf{Durchlauf} durch den Test kann auf einzelne Fehler besser eingegangen werden, da durch die Unteraufgaben die Schwachstellen der Studierenden genauer bestimmt werden können. Ebenfalls soll das Plug-In am Ende des Tests ein kumuliertes, adaptives Feedback geben. Wichtig bei der Umsetzung ist, dass das Plug-In nach seiner Fertigstellung auch von anderen Universitäten genutzt werden kann, weshalb eine umfangreiche Dokumentation gefordert wird.

 	
	\chapter{Qualitätsziele}
	
Im Folgenden legen wir unsere Qualitätsziele dar und erläutern, welche Maßnahmen wir zur Sicherung dieser getroffen haben. Weitere Dokumente wie Vorlagen und Protokolle sind im Anhang zu finden.
Jede Iteration hat eine Länge von zwei Wochen.

        \section{Bedienbarkeit}
Das Plug-In wird nach der Fertigstellung aktiv an der TU Darmstadt im Fachbereich Mathematik verwendet, sowie ggf. von anderen Universitäten. Da die Funktionalität, Fragen zu sortieren und mit Bedingungen zu versehen, über eine GUI bedient wird, muss diese vor allem verständlich, \textbf{aufgeräumt} und intuitiv bedienbar für den Nutzer sein. Auch sollte sie optisch an Moodle angelehnt sein. 
\\
\\
Um das QS-Ziel zu gewährleisten, wollen wir ein Handbuch zu dem Plug-In schreiben. Das Handbuch soll darauf eingehen, wie  das Plug-In zu benutzen ist und was man damit machen kann. Diese wird mit dem Plug-In als Paket mitgeliefert. Zu finden ist sie dann auf Github als Manual.md und auf der moodle.org Plug-In Seite als einfache HTML Anleitung.
\\
\\
Da die Funktionalität des Plugins über das Front-End bestimmt wird, ist es wichtig, dass das Plug-In für den Nutzer einfach zu verwenden ist. Im Rahmen dessen werden wir eine Nutzerstudie mit Einarbeitungszeit in das Plug-In durchführen. Dafür werden wir Ende Januar/Anfang Februar 2017 5-7 Tutoren am Fachbereich Mathematik das Plugin testen lassen. In einem Fragebogen werden diese ihre Eindrücke in Form von Schulnoten (1-6) an uns weitergeben. Sofern aus den Fragebögen eine Zufriedenheit von 75 Prozent (2,5) der Tester erreicht wurde, gehen wir davon aus, dass die Bedienbarkeit nicht mehr geändert werden muss, es sei denn der Auftraggeber möchte diese weiter verbessert haben. Sollte die Befragung zeigen, dass die GUI nicht gut genug ist, wird nachgebessert. Der Auftraggeber entscheidet dann, ob die Qualität seinen Erwartungen entspricht.
\\
Der Fragebogen, welcher zur Sicherstellung der verständlichen, aufgeräumten und intuitiven Bedienbarkeit führen soll, stellt Fragen zur die Aufgeräumtheit der Benutzeroberfläche, die intuitive Bedienung der Benutzeroberfläche und die Qualität des Handbuches.
\\
\\
\\
\\
\\ \\ \\ \\



        \section{Codequalität}
Damit wir das Plugin in Moodle gut integrieren können, müssen wir dessen Richtlinien folgen. Diese Richtlinien sind gegeben durch Styleguides, Namenskonventionen von Klassen und Methoden, sowie Ordnerstrukturen. Links dazu am Ende. Da Moodle international vertreten und Open-Source ist, dokumentieren wir den Quellcode auf Englisch.
\\
\\
Um dies zu gewährleisten verwenden wir Code Reviews mit Checklisten. Jede User-Story muss diese Checkliste erfüllen. Es gibt immer für die jeweilige User Story einen Entwickler und einen Reviewer. Der zufällig aus dem Team gewählte Reviewer prüft mit der Checkliste die User Story und zeigt dem Entwickler der User Story die gefundenen Fehler. Der Entwickler muss bis zur nächsten Iteration die Fehler in seiner Implementierung der User Story behoben haben. 
Die Checkliste baut sich aus folgenden Teilen auf:
\\
\\
1. \hspace{0.1cm}Ist der Style Guide seitens Moodle erfüllt?\\
2. \hspace{0.1cm}Wurde die Namenskonvention seitens Frankenstyle erfüllt?\\
3. \hspace{0.1cm}Ist die erforderliche Ordnerstruktur gegeben ?\\
4. \hspace{0.1cm}Ist die Dokumentation auf Englisch und ausreichend?
\\
\\
Die Codequalität hilft uns und anderen Entwicklern den Code besser verstehen und  
erweitern zu können.\\

Details zu... \\
\textcolor{blue}{\href{https://docs.moodle.org/32/en/MoodleDocs:Style_guide}{StyleGuide}}\\
\textcolor{blue}{\href{https://docs.moodle.org/dev/Frankenstyle}{Frankenstyle}}\\
\textcolor{blue}{\href{https://docs.moodle.org/dev/Activity_modules}{Ordnerstruktur}}\\


	
	\chapter{Projekttagebuch}
	
	Das Plugin zur Ablaufsteuerung von Moodle-Tests wird unter der GNU GPLv3+ Lizenz entwickelt.
	Desweiteren konnte ein Teammitglied aufgrund von familiären Problemen den kompletten Dezember nicht am Projekt mitarbeiten.
			
	
\appendix
	\chapter{Anhang}
		(Am Ende des Projekts nachzureichen)\\
		Beleg für durchgeführte Maßnahmen, bzw. falls nicht durchgeführt eine Begründung wieso die Durchführung nicht möglich oder nicht erfolgt ist. \\
		Weitere Anforderungen sind den Unterlagen und der Vorlesung zur Projektbegleitung zu entnehmen.
	

		
	
\end{document}